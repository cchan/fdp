\documentclass[main.tex]{subfiles}
\begin{document}
    \section{Eddy Current Brakes}
    With only 1.6 kilometers of tube to work with, it’s extremely difficult to dissipate the huge kinetic energy at 100m/s so quickly. Pure friction braking, as with the braking calipers above, cannot work alone, since either the enormous shear forces and heat generated will destroy the brakes, or the pod will not stop within the length of the tube. [NUMBERS?] Eddy current brakes have the advantage of dissipating most of the kinetic energy of the pod into an external heatsink - the I-beams.

    \subsection{Overall Design}
    [Comments on how this integrates with friction braking calipers]\\
    Full labeled detailed CAD\\
    Mass, power, etc.\\
    How is it failsafe?\\
    Full cost breakdown, comments on manufacturability and production costs\\
    “A full Bill of Materials can be found in Appendix C.”

    \subsection{Physical Modeling}
    Fun magnet graphs!!\\
    Comparison of Force Velocity Graphs and A-t V-t D-t\\
    Thermal - I beam heat generation, show it will not be harmful

    \subsection{Structural}
    How is it failsafe?\\
    What are several reasonable and edge-case loading scenarios, and how does the FEA look for all of those? Justify your “reasonable” scenarios. If possible to simulate, how many cycles might it withstand?\\
    How will we deal with imperfections and irregularities in the track, both lateral and vertical? (Simulation would be good)

    \subsection{Tests \& Validation (completed and planned)}
    Test rig (see Next Steps as well)\\
    Fault tolerance, potential failure modes (FMEA)

    \subsection{Safety and handling}
    Magnet safety and handling procedures
    How do we verify that the magnets are magnetized enough? How do we preserve the magnetism?
\end{document}
