\documentclass[main.tex]{subfiles}
\begin{document}
    \chapter*{Quick Reference}
    \begin{enumerate}
        \item Description of team and updated list of all associated team members and advisors
        \item Design description for Pod. At a minimum, this should include:
        \begin{enumerate}
            \item Pod top-level design summary\flushright{\pageref{ch:top-level-design}}
            \item Pod dimensions
            \item Pod mass by subsystem
            \item Pod payload capability
            \item Pod materials
            \item Pod power source and consumption
            \item Pod navigation mechanism
            \item Pod levitation mechanism (if any)
            \item Pod propulsion mechanism
            \item Pod braking mechanism
            \item Pod stability mechanisms (e.g. attitude and lateral motion)
            \item Pod aerodynamic coefficients
            \item Pod magnetic parameters (if applicable)
        \end{enumerate}
        \item Predicted Pod thermal profile
        \item Predicted Pod trajectory (speed versus distance)
        \item Predicted vibration environments
        \item Pod structural design cases: at a minimum, this shall include initial acceleration, nominal deceleration, and a reasonably foreseeable off-nominal crash
        \item Pod production schedule
        \item Pod cost breakdown
        \item Sensor list and location map
        \item Comments on scalability to an operational Hyperloop with respect to:
        \begin{enumerate}
            \item System size (increased tube length, tube diameter, and Pod size)
            \item Cost (both production and maintenance)
            \item Estimated Pod mass and cost if built full-scale
            \item Maintenance (e.g. not requiring specialized alignment tools, hot-swappable subsystems)
        \end{enumerate}
        \item Loading and unloading plan
        \begin{enumerate}
            \item Full descriptions of all functional tests (see Sections 10 and 12)
            \item Full description of Ready-to-Launch checklist/state (e.g. Loop Computer in “Launch Mode” and sending telemetry, Pod hovering at 0.25 inches)
            \item Full description of Ready-to-Remove checklist/state (e.g. Wheels locked, Power Off)
            \item Description of how Pod is moved from Staging Area to Hyperloop
            \item Description of how Pod is moved from Hyperloop to Exit Area
        \end{enumerate}
        \item List and description of any stored energy on the Pod (i.e. pressure vessels, batteries)
        \item List of any hazardous materials, if any
        \item Description of safety features including:
        \begin{enumerate}
            \item Hardware and software inhibits on braking during the acceleration phase
            \item Mechanisms to mitigate a complete loss of Pod power
            \item Pod robustness to a tube breach resulting in rapid pressurization
            \item Fault tolerance of braking, levitation, and other subsystems
            \item Single point of failures within the Pod
            \item Recovery plan if Pod becomes immovable within tube
            \item Implementation of the Pod-Stop command
        \end{enumerate}
        \item Component and system test program before the Pod arrives for the Competition
        \item Vacuum Compatibility Analysis
        \item If a returning Pod, highlight the modifications and upgrades made
    \end{enumerate}

    \addcontentsline{toc}{chapter}{Quick Reference}
\end{document}